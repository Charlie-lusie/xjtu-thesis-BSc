\documentclass[truetimes,
               print, 
               timesmath
              ]{xjtubsc}

% truetimes 选项会使用windows 自带的 Times New Roman 字体,否则使用tex发行版的times字体,几乎没有区别,除了专业人士看不出差别。开启 truetiems 选项必须使用 XeLaTeX 编译。否则请删除truetiems。推荐开启。

% print 选项会去除超链接的颜色。以供打印。根据需求开启。

% timesmath 将会使用times风格的数学字体,以使与正文的times英文字体更加搭配。根据需求开启。

\usepackage{blindtext}
\usepackage{tikz-cd}
\DeclareMathOperator{\coker}{coker}
\usetikzlibrary{matrix, calc, arrows}

\usepackage{graphicx}
\usepackage{subfigure} 
% Comments by hy_haoyun
% Note the subfigure is old-fashioned.
% We'd better use subfig instead.
% But I don't have enough time to modify the example.
\usepackage{booktabs}
\usepackage{listings}
\usepackage{xcolor}
\lstset{upquote,
	keepspaces=true,
	columns=spaceflexible,
	basicstyle=\ttfamily\footnotesize,%
	breaklines=true,
	breakindent=0pt,
	xleftmargin=0pt,
	xrightmargin=6pt,%
	language=matlab,
}
\graphicspath{{figures/}}

% 已知问题:
%	1. 关键词与摘要之间的间距不太合适。
%	2. 	页眉距定端距离有问题



\begin{document}


%================================================
%	下面填写基本信息
%================================================


% 以下三项是必须的。
\author{小和尚}{xiao he shang}	%	作者{中文}{英文}
\title{小和尚犯色戒之检讨}{I'm sorry}	%	题目{中文}{英文}
\advisor{老和尚}{lao he shang}	%	导师{中文}{英文}

% 如果不希望 \LaTeX{} 生成任务书、评议书、意见书、答辩结果,下面的信息可以不填。

\college{神学}	%	学院
\department{佛学}	%	专业(系)
\class{禅 91}	%	班级
\place{大渣交}	%	毕设地点
\beginyear{2012}	%	开始年
\beginmonth{12}	%	开始月
\begindate{20}	%	开始日
\endyear{2012}	%	结束年
\endmonth{12}	%	结束月
\enddate{21}	%	结束日

\begin{INFObackground} % 课题的背景、意义及培养目标 
世人都晓神仙好,惟有功名忘不了!
古今将相在何方?荒冢一堆草没了.
世人都晓神仙好,只有金银忘不了!
终朝只恨聚无多,及到多时眼闭了.
世人都晓神仙好,只有姣妻忘不了!
君生日日说恩情,君死又随人去了.
世人都晓神仙好,只有儿孙忘不了!
痴心父母古来多,孝顺儿孙谁见了?
\end{INFObackground}

\begin{INFOdata} % 设计(论文)的原始数据与资料
陋室空堂,当年笏满床,衰草枯杨,曾为歌舞场.
蛛丝儿结满雕梁,绿纱今又糊在蓬窗上.
说什么脂正浓,粉正香,如何两鬓又成霜?
昨日黄土陇头送白骨,今宵红灯帐底卧鸳鸯.
金满箱,银满箱,展眼乞丐人皆谤.
正叹他人命不长,那知自己归来丧!
训有方,保不定日后作强梁.
择膏粱,谁承望流落在烟花巷!
因嫌纱帽小,致使锁枷杠,昨怜破袄寒,今嫌紫蟒长:
乱烘烘你方唱罢我登场,反认他乡是故乡.
甚荒唐,
到头来都是为他人作嫁衣裳!
\end{INFOdata}

\begin{INFOtask} % 课题的主要任务
无故寻愁觅恨,有时似傻如狂.
纵然生得好皮囊,腹内原来草莽.
潦倒不通世务,愚顽怕读文章.
行为偏僻性乖张,那管世人诽谤!

富贵不知乐业, 贫穷难耐凄凉.
可怜辜负好韶光,于国于家无望.
天下无能第一,古今不肖无双.
寄言纨绔与膏粱:莫效此儿形状!
\end{INFOtask}

\begin{INFOrequirement} % 课题的主要任务
斜阳寒草带重门,苔翠盈铺雨后盆.
玉是精神难比洁,雪为肌骨易销魂.
芳心一点娇无力,倩影三更月有痕.
莫谓缟仙能羽化,多情伴我咏黄昏.
\end{INFOrequirement}

\begin{INFOsubmit} % 课题的主要任务
半卷湘帘半掩门,碾冰为土玉为盆.
偷来梨蕊三分白,借得梅花一缕魂.
月窟仙人缝缟袂,秋闺怨女拭啼痕.
娇羞默默同谁诉,倦倚西风夜已昏.
\end{INFOsubmit}

\begin{INFOreference} % 课题的主要任务
珍重芳姿昼掩门,自携手瓮灌苔盆.
胭脂洗出秋阶影,冰雪招来露砌魂.
淡极始知花更艳,愁多焉得玉无痕.
欲偿白帝凭清洁,不语婷婷日又昏.
\end{INFOreference}

%================================================
%	正文前
%================================================
\frontmatter

\extrapages % 用于输出任务书、评议书、意见书、答辩结果。如果不希望 \LaTeX{} 生成任务书、评议书、意见书、答辩结果,可以注释掉此句。

\begin{abstractcn} % 中文摘要
桃花帘外东风软,桃花帘内晨妆懒.
帘外桃花帘内人,人与桃花隔不远.
东风有意揭帘栊,花欲窥人帘不卷.
桃花帘外开仍旧,帘中人比桃花瘦.
花解怜人花也愁,隔帘消息风吹透.
风透湘帘花满庭,庭前春色倍伤情.
闲苔院落门空掩,斜日栏杆人自凭.
凭栏人向东风泣,茜裙偷傍桃花立.
桃花桃叶乱纷纷,花绽新红叶凝碧.
雾裹烟封一万株,烘楼照壁红模糊.
天机烧破鸳鸯锦,春酣欲醒移珊枕.
侍女金盆进水来,香泉影蘸胭脂冷.
胭脂鲜艳何相类,花之颜色人之泪,
若将人泪比桃花,泪自长流花自媚.
泪眼观花泪易干,泪干春尽花憔悴.
憔悴花遮憔悴人,花飞人倦易黄昏.
一声杜宇春归尽,寂寞帘栊空月痕!
\end{abstractcn}

\keywordscn{中文;摘要;关键词} % 中文关键词

\begin{abstracten} % 中文英文
\Blindtext[2][1] % 测试文字
\end{abstracten}

\keywordsen{English; Abstract; Key Words; XJTU} % 英文关键词

\tableofcontents % 生成目录


%================================================
%	正文
%================================================
\mainmatter

\section{绪论}
这是绪论。绪,绪,绪论的绪,论,论,绪论的论。
\subsection{二级标题一}
\blindtext % 测试文字
\subsection{二级标题二}
\blinditemize % 测试文字
\subsubsection{三级标题}
三级哦,亲!

\section{公式测试}
这是公式测试。公,公,公式的公,式,式,公式的式。
\newcommand\abs[1]{\left\lvert#1\right\rvert}
\newcommand*\diff{\mathop{}\!\mathrm{d}}
\begin{align}
\begin{split}\abs{I_1} &= \left\lvert \int_\Omega gRu\diff\Omega\right\rvert\\
  & \le C_3\left[\int_\Omega\left(\int_{a}^x g(\xi,t)\diff\xi\right)^2\diff\Omega\right]^{1/2}
       \times\left[\int_\Omega\left\{u^2_x+\frac{1}{k}
    \left(\int_{a}^x cu_t\diff\xi\right)^2\right\}\diff\Omega\right]^{1/2}\\
  & \le C_4\left\lvert \left\lvert f\left\lvert \widetilde{S}^{-1,0}_{a,-}W_2(\Omega,\Gamma_l)
    \right\rvert\right\rvert\left\lvert \abs{u}\overset{\circ}\to W_2^{\widetilde{A}}
    (\Omega;\Gamma_r,T)\right\rvert\right\rvert.
\end{split}\\
\begin{split}\abs{I_2} &= \left\lvert \int_{0}^T \psi(t)\left\{u(a,t)
  -\int_{\gamma(t)}^a\frac{\diff\theta}{k(\theta,t)}
  \int_{a}^\theta c(\xi)u_t(\xi,t)\diff\xi\right\}\diff t\right\rvert\\
  & \le C_6\left\lvert \left\lvert f\int_\Omega\left\lvert \widetilde{S}^{-1,0}_{a,-}
    W_2(\Omega,\Gamma_l)\right\rvert\right\rvert\left\lvert \abs{u}\overset{\circ}
    \to W_2^{\widetilde{A}}(\Omega;\Gamma_r,T)\right\rvert\right\rvert.
\end{split}\tag{\theequation$'$}
\end{align}

\begin{tikzpicture}[>=triangle 60]
  \matrix[matrix of math nodes,column sep={60pt,between origins},row
    sep={60pt,between origins},nodes={asymmetrical rectangle}] (s)
  {
    &|[name=ka]| \ker f &|[name=kb]| \ker g &|[name=kc]| \ker h \\
    %
    &|[name=A]| A' &|[name=B]| B' &|[name=C]| C' &|[name=01]| 0 \\
    %
    |[name=02]| 0 &|[name=A']| A &|[name=B']| B &|[name=C']| C \\
    %
    &|[name=ca]| \coker f &|[name=cb]| \coker g &|[name=cc]| \coker h \\
  };
  \draw[->] (ka) edge (A)
            (kb) edge (B)
            (kc) edge (C)
            (A) edge (B)
            (B) edge node[auto] {\(p\)} (C)
            (C) edge (01)
            (A) edge node[auto] {\(f\)} (A')
            (B) edge node[auto] {\(g\)} (B')
            (C) edge node[auto] {\(h\)} (C')
            (02) edge (A')
            (A') edge node[auto] {\(i\)} (B')
            (B') edge (C')
            (A') edge (ca)
            (B') edge (cb)
            (C') edge (cc)
  ;
  \draw[->,gray] (ka) edge (kb)
                 (kb) edge (kc)
                 (ca) edge (cb)
                 (cb) edge (cc)
  ;
  \draw[->,gray,rounded corners] (kc) -| node[auto,text=black,pos=.7]
    {\(\partial\)} ($(01.east)+(.5,0)$) |- ($(B)!.35!(B')$) -|
    ($(02.west)+(-.5,0)$) |- (ca);
\end{tikzpicture}


äöüßÄÖÜ

\section{图表测试}
这是图表测试。图,图,图表的图,表,表,图表的表。
\subsection{二级标题1}
\subsection{二级标题2}
\subsubsection{三级标题}
三级哦,亲!



%================================================
%	正文后
%================================================
\backmatter


\mybibliography{ref/refs}


\begin{lstlisting}
\end{lstlisting}

\appendixs{易筋经}
\blindmathpaper
\blindmathpaper
\appendixs{易筋经}



%如果用 pdflatex编译,代码38行报错。应该是分页导致的,
%这个问题急需解决。根本原因应该在于T1的字体编码。

\begin{mydenotation}
\item[HPC] 高性能计算~(High Performance Computing)
\item[cluster] 集群
\item[Itanium] 安腾
\item[SMP] 对称多处理
\item[API] 应用程序编程接口
\item[PI]	聚酰亚胺
\item[MPI]	聚酰亚胺模型化合物,N-苯基邻苯酰亚胺
\item[PBI]	聚苯并咪唑
\item[MPBI]	聚苯并咪唑模型化合物,N-苯基苯并咪唑
\item[PY]	聚吡咙
\item[PMDA-BDA]	均苯四酸二酐与联苯四胺合成的聚吡咙薄膜
\item[$\Delta G$]  	活化自由能~(Activation Free Energy)
\item [$\chi$] 传输系数~(Transmission Coefficient)
\item[$E$] 能量
\item[$m$] 质量
\item[$c$] 光速
\item[$P$] 概率
\item[$T$] 时间
\item[$v$] 速度
\item[劝  学] 君子曰:学不可以已。青,取之于蓝,而青于蓝;冰,水为之,而寒于水。
  木直中绳。(车柔)以为轮,其曲中规。虽有槁暴,不复挺者,(车柔)使之然也。故木
  受绳则直, 金就砺则利,君子博学而日参省乎己,则知明而行无过矣。吾尝终日而思
  矣,  不如须臾之所学也;吾尝(足齐)而望矣,不如登高之博见也。登高而招,臂非加
  长也,  而见者远;  顺风而呼,  声非加疾也,而闻者彰。假舆马者,非利足也,而致
  千里;假舟楫者,非能水也,而绝江河,  君子生非异也,善假于物也。积土成山,风雨
  兴焉;积水成渊,蛟龙生焉;积善成德,而神明自得,圣心备焉。故不积跬步,无以至千
  里;不积小流,无以成江海。骐骥一跃,不能十步;驽马十驾,功在不舍。锲而舍之,朽
  木不折;  锲而不舍,金石可镂。蚓无爪牙之利,筋骨之强,上食埃土,下饮黄泉,用心
  一也。蟹六跪而二螯,非蛇鳝之穴无可寄托者,用心躁也。
\end{mydenotation}


\begin{acknowledgment}
感谢党,感谢国家,感谢蛤蛤。
\end{acknowledgment}



\end{document} 