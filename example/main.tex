%================================================
%	载入文档类,配置文档类选项
%================================================

\documentclass[%truetimes,
               print, 
               %timesmath
              ]{xjtubsc}

% truetimes 选项会使用windows 自带的 Times New Roman 字体,否则使用 TeX 发行版的 Times 字体,几乎没有区别,肉眼凡胎看不出差别。开启 truetiems 选项必须使用 XeLaTeX 编译。否则关闭 truetiems。推荐开启。且推荐使用 XeLaTeX 编译。

% print 选项会去除超链接的颜色。以供打印。根据需求开启。

% timesmath 将会使用 Times 风格的数学字体,以使与正文的 Times 英文字体更加搭配。根据需求开启。

%================================================
%	加载宏包、配置宏包选项、自定义命令等
%================================================


\usepackage{xjtupackages} % 这是对常用宏包的一个打包,供初级用户使用。如需在此基础上载入其他宏包,请先阅读 xjtupackages.sty 文件,以免造成冲突。

\graphicspath{{figures/},{figs/}} % 仿照此格式设置图片路径,需要graphicx支持(xjtupackages中已载入)

\DeclareMathOperator{\coker}{coker}
\newcommand\abs[1]{\left\lvert#1\right\rvert}
\newcommand*\diff{\mathop{}\!\mathrm{d}}


\begin{document}
%================================================
%	下面填写基本信息
%================================================

% 以下三项是必须的。
\author{王鸿鑫、秦雨果、郝运}{Hongxin WANG, Yuguo QIN, Yun HAO}	%	作者{中文}{英文}
\title{西安交通大学理学学士毕业论文 \LaTeX{} 模板示例文件}{Example of \LaTeX{} Template for Bachelor of Science Thesis in XJTU}	%	题目{中文}{英文}
\advisor{王鸿鑫、秦雨果、郝运}{Hongxin WANG, Yuguo QIN, Yun HAO}	%	导师{中文}{英文}

% 如果不希望 \LaTeX{} 生成任务书、评议书、意见书、答辩结果,下面的信息可以不填。
\college{数学与统计}	%	学院
\department{数学}	%	专业(系)
\class{理科数学 91}	%	班级
\place{西安交通大学}	%	毕设地点
\bdate{2012/12/21}	% 开始时间,格式为:年/月/日
\edate{2013/2/15}	% 结束时间,格式为:年/月/日

\begin{INFObackground} % 课题的背景、意义及培养目标 


\end{INFObackground}

\begin{INFOdata} % 设计(论文)的原始数据与资料


\end{INFOdata}

\begin{INFOtask} % 课题的主要任务


\end{INFOtask}

\begin{INFOrequirement} % 课题的主要任务


\end{INFOrequirement}

\begin{INFOsubmit} % 课题的主要任务


\end{INFOsubmit}

\begin{INFOreference} % 课题的主要任务


\end{INFOreference}

%================================================
%	正文前
%================================================
\frontmatter

\extrapages % 用于输出任务书、评议书、意见书、答辩结果。如果不希望 \LaTeX{} 生成任务书、评议书、意见书、答辩结果,可以注释掉此句。

\begin{abstractcn} % 中文摘要
这是 \emph{西安交通大学理学学士毕业论文 \LaTeX{} 模板} 的示例文件,目标使用者为有 \LaTeX{} 经验的同学。如果毫无 \LaTeX{} 使用经验,强烈建议认真学习入门教程\footnote{比如:?、?和?}~24~小时后再使用。使用 LaTeX 模板可减少很多排版上的烦恼,但如果使用不当亦会增加痛苦。

本模板尚未得到官方的认可,如有任何关于其是否被认可的问题,请咨询教务员。我们会尽我们最大的努力使得该模板被官方接受。

如有任何关于模板使用的问题,可以向作者咨询,但不一定能及时得到答复。如果需要 24 小时之内答复,可向作者支付 \emph{五毛} 人民币每次(此技术支持对西安交通大学数学拔尖班同学免费)。

\end{abstractcn}

\keywordscn{中文;摘要;关键词} % 中文关键词

\begin{abstracten} % 英文摘要

This is the \emph{\LaTeX{} template for Bachelor of Science Thesis in Xi'an Jiaotong University} with expected users who have some experience with \LaTeX{}. If you have no experience with \LaTeX{}, we SUGGEST that you learn a basic tutorial on \LaTeX{} for more than 24 hours before using the template. Proper use will free you from the worries of typography of your thesis, however, improper use will definitely increasing your pain.

This template has no authorization from the dean's office. If you have any question with respect to the legality of the template, please consult the dean's office of your department or school.

\end{abstracten}

\keywordsen{English; Abstract; Key Words} % 英文关键词

\tableofcontents % 生成目录


%================================================
%	正文
%================================================
\mainmatter


\section{简介}
这是模板。推荐使用 xelatex 编译,可以使用 pdftex 编译。

可以再 windows、linux、Mac OS 正常编译。

作者是。

项目地址是。

汇报bug。

\section{使用指南}
\subsection{准备工作}

安装ctex or miktex,texlive,mactex。参见(附录)。

更新宏包:特别是 xeCJK 及其相关依赖。配图说明更新方法。

如果有特殊需求,安装 texgyremath 等宏包。 

为了正常使用此模板,并避免不必要的麻烦,请完全安装 miktex 套装(或相应的 ctex 套装)或 texlive 套装的最新版,并在线升级\footnote{至少升级 xeCJK 宏包至最新版,如开启timesmath 选项,请确保有 tex-gyre-math 宏包。}。专业用户可不受此约束。

\subsection{文档基本结构}
工作目录结构如下:
\begin{verbatim}
    Working Dir.:
    |  GBT7714-2005NLang-UTF8-mod.bst
    |  main.tex
    |  make.bat
    |  xjtubsc.cls
    |
    |─figures
    |     |─some figures
    |
    |─pages
    |     |─the files
    |
    |─ref
          |─the refs
\end{verbatim}
其中二级目录不是必须的,而且目录名可以根据自己的喜好任选。

文档的基本结构为
\begin{verbatim}
    \documentclass{xjtubsc}
        <导言部分>
    \begin{document}
        <设置基本信息>
        \frontmatter  
            <任务书等、摘要、目录>
        \mainmatter
            <正文部分>
        \backmatter
            <附录、致谢部分>
\end{verbatim}

UTF8 编码!!!
\subsection{文档类及文档类选项}

truetimes 选项会使用windows 自带的 Times New Roman 字体,否则使用 TeX 发行版的 Times 字体,几乎没有区别,肉眼凡胎看不出差别。开启 truetiems 选项必须使用 XeLaTeX 编译。否则关闭 truetiems。推荐开启。且推荐使用 XeLaTeX 编译。

print 选项会去除超链接的颜色。以供打印。根据需求开启。

timesmath 将会使用 Times 风格的数学字体,以使与正文的 Times 英文字体更加搭配。根据需求开启。

\subsection{加载宏包}
xjtupackages.sty

\subsection{填写基本信息,生成任务书等内容}

首先必须\footnote{此处“必须”是指:是为了得到完整的摘要页面(以及可选的任务书等页面),这三项内容是必须的,如果不定义此三项内容,本模板依然正常编译,只会给出警告(Warning),而不会报错(Error)而停止编译,输出结果中保持相应位置处空白。}填写作者信息、导师信息的内容\footnote{这部分内容可以位于 \verb+\begin{document}+ 之前,也可以位于其后。}。命令格式为 \verb|\command{中文}{英文}|,如下所示:
\begin{verbatim}
    \author{作者}{Author}
    \title{题目}{Title}
    \advisor{导师}{Advisor}
\end{verbatim}


如果需要由 \LaTeX{} 生成 \emph{毕业设计(论文)任务书},\emph{毕业设计(论文)考核评议书},\emph{毕业设计(论文)评审意见书} 以及 \emph{毕业设计(论文)答辩结果} 四个表单页面,那么需要 \verb|\college|、\verb|\department|、\verb|\class|、\verb|\place|、\verb|\bdate|、\verb|\edate| 命令来定义基本信息\footnote{同样,这部分内容可以位于 \verb+\begin{document}+ 之前,也可以位于其后。},并通过 \verb|INFObackground|、\verb|INFOdata|、\verb|INFOtask|、\verb|INFOrequirement|、\verb|INFOsubmit| 以及 \verb|INFOsummit| 环境来填写任务书表单。然后使用 \verb|\extrapages| 命令来生成页面。一个简单的例子如下:
\begin{verbatim}
    \college{数学与统计}	%	学院
    \department{数学}	%	专业(系)
    \class{理科数学 91}	%	班级
    \place{西安交通大学}	%	毕设地点
    \bdate{2012/12/21}	% 开始时间,格式为:年/月/日
    \edate{2013/2/15}	% 结束时间,格式为:年/月/日
    
    \begin{INFObackground} % 课题的背景、意义及培养目标 
        这里输入内容
    \end{INFObackground}
    ……
    
    \frontmatter % 设置为正文前格式
    \extrapages % 用输出任务书、评议书、意见书、答辩结果。
\end{verbatim}
特别需要注意的是开始时间和结束时间的格式,另外,上述内容的任何一项均可不写(当然此时相应表单处空白不填)。还需要注意的是,即使不使用 \verb|\extrapages| 来排版任务书等,摘要页面的页码依然从 V \footnote{罗马数字 5。从正文开始采用阿拉伯数字页码,从 1 开始编号.}开始编号。

在定义完\verb|\author|、\verb|\title|、\verb|\advisor| 三项内容之后,开始输入中英文摘要及关键词并生成目录。示例如下:
\begin{verbatim}
    \frontmatter % 设置为正文前格式,如果已经设置则无需此命令。
    \begin{abstractcn} % 中文摘要
        中文摘要
    \end{abstractcn}
    
    \keywordscn{中文;摘要;关键词} % 中文关键词
    
    \begin{abstracten} % 英文摘要
        English Abstract.
    \end{abstracten}
    
    \keywordsen{English; Abstract; Key Words} % 英文关键词
    
    \tableofcontents % 生成目录
\end{verbatim}
注意为了正确生成目录,需要至少两次编译。

至此完成正文之前的部分。

\subsection{正文部分}

正文部分首先以 \verb|\mainmatter| 开始,以设置为正文部分的格式。

文件结构使用 \verb|\section{}|、\verb|\subsection{}|、\verb|\subsubsection{}| 来控制。而且仅支持此三层结构\footnote{当然支持 \verb|\paragraph{}|,但此很少使用,如果需要添加更深层次结构的支持,请自行 hack。模板作者建议合理安排文档结构,只有在非常罕见的情况下,四级、五级结构才是必须的。},不支持 \verb|\part{}| 和 \verb|\chapter{}|。

为了便于编写,推荐将每个 \verb|\section{}| 单独的保存为一个 \verb|.tex| 文件,然后使用 \verb|\include{foobar.tex}| 命令来插入。

注意如果插入的部分在次一级的文件目录里面,例如我们所推荐的
\begin{verbatim}
    \include{pages/foobar.tex}
\end{verbatim}
请注意 \verb|/| 的方向以及扩展名(Windows、Mac OS、Linux 下可能有所不同)。\textbf{(本段需要完善。)}

其余使用方法同基本文档类完全相同。

\subsection{参考文献}

正文完成后首先使用 \verb|\backmatter| 切换至正文后的排版格式。然后使用相应的命令来实现参考文献的排版。推荐使用 \hologo{BibTeX} 排版参考文献。一个例子如下:
\begin{verbatim}
    \backmatter % 切换至正文后版式。

    \nocite{*} % 显示未明显引用的所有文献
    \bibliographystyle{GBT7714-2005NLang-UTF8-mod} 
    % 使用 GB/T7714-2005 标准的文献格式。
    \bibliography{ref/refs} % bib 文件的为 /ref/refs.bib
\end{verbatim}
注意上面使用了 \verb|GBT7714-2005NLang-UTF8-mod.bst|,请确保工作目录下有此文件。如果想使用其他的格式(注意教务处要求),可以根据自己的喜好替换。

当然也可以使用 thebibliography 环境手工排版参考文献,一个示例如下:
\begin{verbatim}
    \begin{thebibliography}{5}
    \bibitem {}
    \end{thebibliography}
\end{verbatim}

可以用相应的工具生成符合格式的 \verb|.bib| 文件,或相应的 thebibliography 环境代码。网上有很多相应的工具,数据库一般情况下可以导出 \verb|.bib| 文件(Export Citation)。

还有一种较新的实现方式是还用 \textsc{Bib}\LaTeX{}/biber,不推荐初级用户使用。
\subsection{附录}

\subsection{高级部分}

\subsubsection{自定义字体}
\subsubsection{自定义宏包}
something

\section{排版效果展示}

\subsection{公式测试}
这是公式测试。公,公,公式的公,式,式,公式的式。

\blindtext

\begin{align}
\begin{split}\abs{I_1} &= \left\lvert \int_\Omega gRu\diff\Omega\right\rvert\\
  & \le C_3\left[\int_\Omega\left(\int_{a}^x g(\xi,t)\diff\xi\right)^2\diff\Omega\right]^{1/2}
       \times\left[\int_\Omega\left\{u^2_x+\frac{1}{k}
    \left(\int_{a}^x cu_t\diff\xi\right)^2\right\}\diff\Omega\right]^{1/2}\\
  & \le C_4\left\lvert \left\lvert f\left\lvert \widetilde{S}^{-1,0}_{a,-}W_2(\Omega,\Gamma_l)
    \right\rvert\right\rvert\left\lvert \abs{u}\overset{\circ}\to W_2^{\widetilde{A}}
    (\Omega;\Gamma_r,T)\right\rvert\right\rvert.
\end{split}\\
\begin{split}\abs{I_2} &= \left\lvert \int_{0}^T \psi(t)\left\{u(a,t)
  -\int_{\gamma(t)}^a\frac{\diff\theta}{k(\theta,t)}
  \int_{a}^\theta c(\xi)u_t(\xi,t)\diff\xi\right\}\diff t\right\rvert\\
  & \le C_6\left\lvert \left\lvert f\int_\Omega\left\lvert \widetilde{S}^{-1,0}_{a,-}
    W_2(\Omega,\Gamma_l)\right\rvert\right\rvert\left\lvert \abs{u}\overset{\circ}
    \to W_2^{\widetilde{A}}(\Omega;\Gamma_r,T)\right\rvert\right\rvert.
\end{split}\tag{\theequation$'$}
\end{align}

\[
\begin{tikzpicture}[>=triangle 60]
  \matrix[matrix of math nodes,column sep={60pt,between origins},row
    sep={60pt,between origins},nodes={asymmetrical rectangle}] (s)
  {
    &|[name=ka]| \ker f &|[name=kb]| \ker g &|[name=kc]| \ker h \\
    %
    &|[name=A]| A' &|[name=B]| B' &|[name=C]| C' &|[name=01]| 0 \\
    %
    |[name=02]| 0 &|[name=A']| A &|[name=B']| B &|[name=C']| C \\
    %
    &|[name=ca]| \coker f &|[name=cb]| \coker g &|[name=cc]| \coker h \\
  };
  \draw[->] (ka) edge (A)
            (kb) edge (B)
            (kc) edge (C)
            (A) edge (B)
            (B) edge node[auto] {\(p\)} (C)
            (C) edge (01)
            (A) edge node[auto] {\(f\)} (A')
            (B) edge node[auto] {\(g\)} (B')
            (C) edge node[auto] {\(h\)} (C')
            (02) edge (A')
            (A') edge node[auto] {\(i\)} (B')
            (B') edge (C')
            (A') edge (ca)
            (B') edge (cb)
            (C') edge (cc)
  ;
  \draw[->,gray] (ka) edge (kb)
                 (kb) edge (kc)
                 (ca) edge (cb)
                 (cb) edge (cc)
  ;
  \draw[->,gray,rounded corners] (kc) -| node[auto,text=black,pos=.7]
    {\(\partial\)} ($(01.east)+(.5,0)$) |- ($(B)!.35!(B')$) -|
    ($(02.west)+(-.5,0)$) |- (ca);
\end{tikzpicture}\]
\blindtext

äöüßÄÖÜ

\subsection{插图、表格}
这是图表测试。图,图,图表的图,表,表,图表的表。
\subsubsection{三级标题}
\subsubsection{三级标题}
\subsubsection{三级标题}
三级哦,亲!

\subsection{定理、定义}

\subsection{图文混排、高级排版}

\subsection{化学排版}

\definesubmol{a}{-P(=[::-90,0.75]O)(-[::90,0.75]HO)-}
\chemfig{[:-54]*5((--[::60]O([::-60]!aO([::-60]!aO([::60]!aHO))))<(-OH)
-[,,,,line width=2pt](-OH)>(-N*5(-=N-*6(-(-NH_2)=N-=N-)=_-))-O-)}

\subsection{绘图}

\subsection{代码、算法}

下面是各种语言书写的 hello world。

C 语言版本:
\begin{minted}{c}
#inlude<stdio.h>
int main(void)	
{
    printf("Hello, World!);
    return 0;
}
\end{minted}

Java 版本
\begin{minted}{java}
public class HelloWorld {
    public static void main(String [] args) {
        System.out.println("Hello World!");
    }
}
\end{minted}

HTML 标记语言
\begin{minted}{html}
<!DOCTYPE html>
<html>
    <body>
        Hello, world!
    </body>
</html>
\end{minted}

Scala  语言:
\begin{minted}{scala}
object HelloWorld extends App {
  println("Hello, world!")
}
\end{minted}

\subsection{bib}

this is some test\cite{IEEE-1363,Krasnogor2004e}
\section{版权声明}
声明

\section{免责声明}
声明

%================================================
%	正文后
%================================================
\backmatter

\nocite{*}
\bibliographystyle{GBT7714-2005NLang-UTF8-mod} % plain
\bibliography{ref/refs}


\appendixs{LaTeX 发行版安装及升级指南}

something here
\appendixs{易筋经}

\begin{mydenotation}
\item[HPC] 高性能计算~(High Performance Computing)
\item[cluster] 集群
\item[Itanium] 安腾
\item[SMP] 对称多处理
\item[API] 应用程序编程接口
\item[PI]	聚酰亚胺
\item[MPI]	聚酰亚胺模型化合物,N-苯基邻苯酰亚胺
\item[PBI]	聚苯并咪唑
\item[MPBI]	聚苯并咪唑模型化合物,N-苯基苯并咪唑
\item[PY]	聚吡咙
\item[PMDA-BDA]	均苯四酸二酐与联苯四胺合成的聚吡咙薄膜
\item[$\Delta G$]  	活化自由能~(Activation Free Energy)
\item [$\chi$] 传输系数~(Transmission Coefficient)
\item[$E$] 能量
\item[$m$] 质量
\item[$c$] 光速
\item[$P$] 概率
\item[$T$] 时间
\item[$v$] 速度
\item[劝  学] 君子曰:学不可以已。青,取之于蓝,而青于蓝;冰,水为之,而寒于水。
  木直中绳。(车柔)以为轮,其曲中规。虽有槁暴,不复挺者,(车柔)使之然也。故木
  受绳则直, 金就砺则利,君子博学而日参省乎己,则知明而行无过矣。吾尝终日而思
  矣,  不如须臾之所学也;吾尝(足齐)而望矣,不如登高之博见也。登高而招,臂非加
  长也,  而见者远;  顺风而呼,  声非加疾也,而闻者彰。假舆马者,非利足也,而致
  千里;假舟楫者,非能水也,而绝江河,  君子生非异也,善假于物也。积土成山,风雨
  兴焉;积水成渊,蛟龙生焉;积善成德,而神明自得,圣心备焉。故不积跬步,无以至千
  里;不积小流,无以成江海。骐骥一跃,不能十步;驽马十驾,功在不舍。锲而舍之,朽
  木不折;  锲而不舍,金石可镂。蚓无爪牙之利,筋骨之强,上食埃土,下饮黄泉,用心
  一也。蟹六跪而二螯,非蛇鳝之穴无可寄托者,用心躁也。
\end{mydenotation}


\begin{acknowledgment}
感谢党,感谢国家,感谢蛤蛤。
\end{acknowledgment}


\end{document} 