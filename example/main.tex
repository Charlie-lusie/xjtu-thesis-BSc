%================================================
%	载入文档类,配置文档类选项
%================================================

\documentclass[%truetimes,
               print, 
               timesmath
              ]{xjtubsc}

% truetimes 选项会使用windows 自带的 Times New Roman 字体,否则使用 TeX 发行版的 Times 字体,几乎没有区别,肉眼凡胎看不出差别。开启 truetiems 选项必须使用 XeLaTeX 编译。否则关闭 truetiems。推荐开启。且推荐使用 XeLaTeX 编译。

% print 选项会去除超链接的颜色。以供打印。根据需求开启。

% timesmath 将会使用 Times 风格的数学字体,以使与正文的 Times 英文字体更加搭配。根据需求开启。

%================================================
%	加载宏包、配置宏包选项、自定义命令等
%================================================

\usepackage{blindtext}
\usepackage[x11names]{xcolor}
\usepackage{tikz-cd}
\DeclareMathOperator{\coker}{coker}
\usetikzlibrary{matrix, calc, arrows}
\usepackage{graphicx}
\graphicspath{{figures/}}
\usepackage{subfig}
\usepackage{booktabs}
\usepackage{listings}
\lstset{upquote,
	keepspaces=true,
	columns=spaceflexible,
	basicstyle=\ttfamily\footnotesize,%
	breaklines=true,
	breakindent=0pt,
	xleftmargin=0pt,
	xrightmargin=6pt,%
	language=matlab,
}
\lstset{extendedchars=false}
% 这一句对于 pdf 编译特别重要,否则如果有代码有分页且\section{}中有中文,则会报错。
% 参考 http://bbs.ctex.org/forum.php?mod=viewthread&tid=43728

\begin{document}
%================================================
%	下面填写基本信息
%================================================

% 以下三项是必须的。
\author{作者}{Author}	%	作者{中文}{英文}
\title{西安交通大学理学学士毕业论文 \LaTeX{} 模板示例文件}{Example of \LaTeX{} Template for Bachelor of Science Thesis in XJTU}	%	题目{中文}{英文}
\advisor{导师}{Advisor}	%	导师{中文}{英文}

% 如果不希望 \LaTeX{} 生成任务书、评议书、意见书、答辩结果,下面的信息可以不填。
\college{数学与统计}	%	学院
\department{数学}	%	专业(系)
\class{理科数学 91}	%	班级
\place{西安交通大学}	%	毕设地点
\bdate{2012/12/21}	% 开始时间,格式为:年/月/日
\edate{2013/2/15}	% 结束时间,格式为:年/月/日

\begin{INFObackground} % 课题的背景、意义及培养目标 


\end{INFObackground}

\begin{INFOdata} % 设计(论文)的原始数据与资料


\end{INFOdata}

\begin{INFOtask} % 课题的主要任务


\end{INFOtask}

\begin{INFOrequirement} % 课题的主要任务


\end{INFOrequirement}

\begin{INFOsubmit} % 课题的主要任务


\end{INFOsubmit}

\begin{INFOreference} % 课题的主要任务


\end{INFOreference}

%================================================
%	正文前
%================================================
\frontmatter

\extrapages % 用于输出任务书、评议书、意见书、答辩结果。如果不希望 \LaTeX{} 生成任务书、评议书、意见书、答辩结果,可以注释掉此句。

\begin{abstractcn} % 中文摘要
这是 \emph{西安交通大学理学学士毕业论文 \LaTeX{} 模板} 的示例文件。推荐有 \LaTeX{} 经验的童鞋使用。如果毫无 \LaTeX{} 使用经验,可以先学习 24 小时入门教程再使用。此模板可以减轻排版的痛苦但如果使用不但,亦会增加痛苦。

如有任何问题,向作者咨询,但不一定能即使得到答复。如果需要 24 小时之内答复,可向作者支付 \emph{五毛} 人民币每次。

\end{abstractcn}

\keywordscn{五毛;摘要;关键词} % 中文关键词

\begin{abstracten} % 英文摘要
\Blindtext[2][1] % 测试文字

\end{abstracten}

\keywordsen{English; Abstract; Key Words; XJTU} % 英文关键词

\tableofcontents % 生成目录


%================================================
%	正文
%================================================
\mainmatter

\section{绪论}
这是绪论。绪,绪,绪论的绪,论,论,绪论的论。
\subsection{二级标题一}
\blindtext % 测试文字
\subsection{二级标题二}
\blinditemize % 测试文字
\subsubsection{三级标题}
三级哦,亲!

\section{公式测试}
这是公式测试。公,公,公式的公,式,式,公式的式。
\newcommand\abs[1]{\left\lvert#1\right\rvert}
\newcommand*\diff{\mathop{}\!\mathrm{d}}
\begin{align}
\begin{split}\abs{I_1} &= \left\lvert \int_\Omega gRu\diff\Omega\right\rvert\\
  & \le C_3\left[\int_\Omega\left(\int_{a}^x g(\xi,t)\diff\xi\right)^2\diff\Omega\right]^{1/2}
       \times\left[\int_\Omega\left\{u^2_x+\frac{1}{k}
    \left(\int_{a}^x cu_t\diff\xi\right)^2\right\}\diff\Omega\right]^{1/2}\\
  & \le C_4\left\lvert \left\lvert f\left\lvert \widetilde{S}^{-1,0}_{a,-}W_2(\Omega,\Gamma_l)
    \right\rvert\right\rvert\left\lvert \abs{u}\overset{\circ}\to W_2^{\widetilde{A}}
    (\Omega;\Gamma_r,T)\right\rvert\right\rvert.
\end{split}\\
\begin{split}\abs{I_2} &= \left\lvert \int_{0}^T \psi(t)\left\{u(a,t)
  -\int_{\gamma(t)}^a\frac{\diff\theta}{k(\theta,t)}
  \int_{a}^\theta c(\xi)u_t(\xi,t)\diff\xi\right\}\diff t\right\rvert\\
  & \le C_6\left\lvert \left\lvert f\int_\Omega\left\lvert \widetilde{S}^{-1,0}_{a,-}
    W_2(\Omega,\Gamma_l)\right\rvert\right\rvert\left\lvert \abs{u}\overset{\circ}
    \to W_2^{\widetilde{A}}(\Omega;\Gamma_r,T)\right\rvert\right\rvert.
\end{split}\tag{\theequation$'$}
\end{align}

\begin{tikzpicture}[>=triangle 60]
  \matrix[matrix of math nodes,column sep={60pt,between origins},row
    sep={60pt,between origins},nodes={asymmetrical rectangle}] (s)
  {
    &|[name=ka]| \ker f &|[name=kb]| \ker g &|[name=kc]| \ker h \\
    %
    &|[name=A]| A' &|[name=B]| B' &|[name=C]| C' &|[name=01]| 0 \\
    %
    |[name=02]| 0 &|[name=A']| A &|[name=B']| B &|[name=C']| C \\
    %
    &|[name=ca]| \coker f &|[name=cb]| \coker g &|[name=cc]| \coker h \\
  };
  \draw[->] (ka) edge (A)
            (kb) edge (B)
            (kc) edge (C)
            (A) edge (B)
            (B) edge node[auto] {\(p\)} (C)
            (C) edge (01)
            (A) edge node[auto] {\(f\)} (A')
            (B) edge node[auto] {\(g\)} (B')
            (C) edge node[auto] {\(h\)} (C')
            (02) edge (A')
            (A') edge node[auto] {\(i\)} (B')
            (B') edge (C')
            (A') edge (ca)
            (B') edge (cb)
            (C') edge (cc)
  ;
  \draw[->,gray] (ka) edge (kb)
                 (kb) edge (kc)
                 (ca) edge (cb)
                 (cb) edge (cc)
  ;
  \draw[->,gray,rounded corners] (kc) -| node[auto,text=black,pos=.7]
    {\(\partial\)} ($(01.east)+(.5,0)$) |- ($(B)!.35!(B')$) -|
    ($(02.west)+(-.5,0)$) |- (ca);
\end{tikzpicture}


äöüßÄÖÜ

\section{图表测试}
这是图表测试。图,图,图表的图,表,表,图表的表。
\subsection{二级标题1}
\subsection{二级标题2}
\subsubsection{三级标题}
三级哦,亲!



%================================================
%	正文后
%================================================
\backmatter


\mybibliography{ref/refs}



\begin{lstlisting}
close all;
clc;
clear all;
lam=10/20^2;
eps=0.00001;
p=1/2;
I=imread('001.bmp');
J=I;
MASK=J;
[width,height] = size(MASK);
U=double(J);
V=double(J);
T=double(J);
n=1;
IterTimes=100;
p=1/2;
I=imread('001.bmp');
J=I;
MASK=J;
[width,height] = size(MASK);
U=double(J);
V=double(J);
T=double(J);
n=1;
IterTimes=100;p=1/2;
I=imread('001.bmp');
J=I;
MASK=J;
[width,height] = size(MASK);
U=double(J);
V=double(J);
T=double(J);
n=1;
IterTimes=100;
while n<=IterTimes
for i=2:(width-1)
for j=2:(height-1)
gridUe2=(eps^2+(V(i+1,j)-V(i,j))^2+
         (1.0/4)*(V(i+1,j+1)-V(i+1,j-1))^2)^((2-p)/2);
gridUw2=(eps^2+(V(i,j)-V(i-1,j))^2+
         (1.0/4)*(V(i-1,j+1)-V(i-1,j-1))^2)^((2-p)/2);
gridUn2=(eps^2+(V(i,j)-V(i,j+1))^2+
         (1.0/4)*(V(i+1,j+1)-V(i-1,j+1))^2)^((2-p)/2);
gridUs2=(eps^2+(V(i,j)-V(i,j-1))^2+
         (1.0/4)*(V(i+1,j-1)-V(i-1,j-1))^2)^((2-p)/2);
we=1/gridUe2;
ww=1/gridUw2;
wn=1/gridUn2;
ws=1/gridUs2;
sum=we+ww+wn+ws;
h0=lam/(lam+sum);
U(i,j)=(ww*V(i-1,j)+we*V(i+1,j)+wn*V(i,j+1)+
		ws*V(i,j-1))/(lam+sum)+h0*T(i,j);
V(i,j)=U(i,j);
end
end
n=n+1;
V=U;
end
V=U;
figure;
imshow(uint8(V));
\end{lstlisting}

\appendixs{易筋经}
\blindmathpaper
\blindmathpaper
\appendixs{易筋经}

\begin{mydenotation}
\item[HPC] 高性能计算~(High Performance Computing)
\item[cluster] 集群
\item[Itanium] 安腾
\item[SMP] 对称多处理
\item[API] 应用程序编程接口
\item[PI]	聚酰亚胺
\item[MPI]	聚酰亚胺模型化合物,N-苯基邻苯酰亚胺
\item[PBI]	聚苯并咪唑
\item[MPBI]	聚苯并咪唑模型化合物,N-苯基苯并咪唑
\item[PY]	聚吡咙
\item[PMDA-BDA]	均苯四酸二酐与联苯四胺合成的聚吡咙薄膜
\item[$\Delta G$]  	活化自由能~(Activation Free Energy)
\item [$\chi$] 传输系数~(Transmission Coefficient)
\item[$E$] 能量
\item[$m$] 质量
\item[$c$] 光速
\item[$P$] 概率
\item[$T$] 时间
\item[$v$] 速度
\item[劝  学] 君子曰:学不可以已。青,取之于蓝,而青于蓝;冰,水为之,而寒于水。
  木直中绳。(车柔)以为轮,其曲中规。虽有槁暴,不复挺者,(车柔)使之然也。故木
  受绳则直, 金就砺则利,君子博学而日参省乎己,则知明而行无过矣。吾尝终日而思
  矣,  不如须臾之所学也;吾尝(足齐)而望矣,不如登高之博见也。登高而招,臂非加
  长也,  而见者远;  顺风而呼,  声非加疾也,而闻者彰。假舆马者,非利足也,而致
  千里;假舟楫者,非能水也,而绝江河,  君子生非异也,善假于物也。积土成山,风雨
  兴焉;积水成渊,蛟龙生焉;积善成德,而神明自得,圣心备焉。故不积跬步,无以至千
  里;不积小流,无以成江海。骐骥一跃,不能十步;驽马十驾,功在不舍。锲而舍之,朽
  木不折;  锲而不舍,金石可镂。蚓无爪牙之利,筋骨之强,上食埃土,下饮黄泉,用心
  一也。蟹六跪而二螯,非蛇鳝之穴无可寄托者,用心躁也。
\end{mydenotation}


\begin{acknowledgment}
感谢党,感谢国家,感谢蛤蛤。
\end{acknowledgment}


\end{document} 