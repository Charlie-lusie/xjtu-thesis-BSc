%============================%
\college{数学与统计}
\department{数学}
\class{理科数学91}
\student{疯狂的}
\yourtitle{交大毕设模板}
\place{我大渣交}
\beginyear{2012}
\beginmonth{1}
\begindate{1}
\endyear{2013}
\endmonth{12}
\enddate{30}



%==================毕业设计(论文)任务书  ===================%



%修改某些距离
\parindent=0pt
%\setlength\parskip{1ex}
\renewcommand{\baselinestretch}{1.6}\selectfont
%注意这里必须加上\selectfont 使得行距设置生效,放在cls里面就不用这个命令了。

% 这里缺少题头,利用minipage结合表格来制作。
%字距得调一下
\begin{minipage}[b]{1\linewidth} 
\hskip 2cm
\begin{tabular*}{0.3\linewidth}{lp{5cm}}
{\ziju{0.5}{\bfseries \LARGE 西安交通大学 }}
\end{tabular*}
\hskip 2cm
\begin{tabular*}{0.5\linewidth}{lp{6cm}}
{\ziju{2.2}系(所)}\myunderline{  \hfill}\\
系(所)主任\myunderline{ \hfill}\\
{\ziju{0.7}批准日期 }\myunderline{ \hfill}\\ 
\end{tabular*}
 \end{minipage}
 
\vspace{0.9em}


\begin{center}\bfseries\huge % 具体的字号我不清楚,需要改一下
毕业设计(论文)任务书   
 % 这里不是宋体,但是应该没人管。我希望改的不依赖于xetex,暂时先这样。不是核心问题。加到bug里面,或者谁改一下?
\vspace{1em}
\end{center}
\sectionmark{\small 毕业设计(论文)任务书}
\par\myunderline{\@college}~院~\myunderline{\@department}~系~\myunderline{\@class}~班\hspace{\ccwd}学生\CJKunderline{\hfill \@student \hfill}

\par 毕业设计(论文)课题\hspace{1ex}\CJKunderline{\hfill \@yourtitle \hfill}

% 这里还少一行,谁来补一下!
% hongxin:已补
\par 毕业设计(论文)工作自~\myunderline[3]{\@beginyear}~年~\myunderline[2]{\@beginmonth}~月~\myunderline[1.5]{\@begindate}~日起至\myunderline[3]{\@endyear}~年~\myunderline[2]{\@endmonth}~月~\myunderline[1.5]{\@enddate}~日止

\par 毕业设计(论文)进行地点\hspace{1ex}\CJKunderline{\hfill \@place \hfill}

% 上面的这些东西可以通过设定参数 c l 等来决定居中还是左对齐,还可以通过字距\ziju来设定分散开来。这个可以先不实现。不是非常必要的。

% 把上面所有的空设置成像 \title一些样的命令,自动带进去。这个鸿鑫解决如何?
% hongxin:已解决





% 下面的东西的通过通过定义环境来实现:

% 环境有如下的参数
% #1 预定的行数,即\ifnum\hangshu<5 中的 5 这个值。给个默认值5或者 6
% #2 题头,如 课题的背景、意义及培养目标
% 环境开始从为
%	\par #2 \par\CJKunderline*{%\hskip2\ccwd
% 环境结束为
%	\hfill}
%	\par\ifnum\prevgraf<#1 
%	\newcount\hangshu 
%	\hangshu=\prevgraf
%	\loop
%		\par\noindent\CJKunderline{\hfill}
%		\advance\hangshu 1 
%	\ifnum\hangshu<#1
%	\repeat
%	\fi
% 注意保留上面的\hfill},不要\hskip2\ccwd,但是在调用环境的时候在内容前加上\hskip2\ccwd(比如这一份表格的情况),或者不加(比如下个表格的情况)
\begin{ulfield2}[5]{课题的背景、意义及培养目标}
第一部分,机器学习的底层理论:机器学习的底层理论有一些,比如推理与规划、近似可计算理论、正则化、提升理论、核方法、当然还有大名鼎鼎的统
\end{ulfield2}

\begin{ulfield2}[6]{设计(论文)的原始数据与资料}
第一部分,机器学习的底层理论:机器学习的底层理论有一些,比如推理与规划、近似可计算理论、正则化、提升理论、核方法、当然还有大名鼎鼎的统
\end{ulfield2}

\begin{ulfield2}[7]{课题的主要任务}
第一部分,机器学习的底层理论:机器学习的底层理论有一些,比如推理与规划、近似可计算理论、正则化、提升理论、核方法、当然还有大名鼎鼎的统
\end{ulfield2}

\begin{ulfield2}[7]{课题的基本要求(工程设计类题应有技术经济分析要求)}
第一部分,机器学习的底层理论:机器学习的底层理论有一些,比如推理与规划、近似可计算理论、正则化、提升理论、核方法、当然还有大名鼎鼎的统第一部分,机器学习的底层理论:机器学习的底层理论有一些,比如推理与规划、近似可计算理论、正则化、提升理论、核方法、当然还有大名鼎鼎的统第一部分,机器学习的底层理论:机器学习的底层理论有一些,比如推理与规划、近似可计算理论、正则化、提升理论、核方法、当然还有大名鼎鼎的统
\end{ulfield2}

\begin{ulfield2}[6]{完成任务后提交的书面材料要求(图纸规格、数量,论文字数,外文翻译字数等)}
第一部分,机器学习的底层理论:机器学习的底层理论有一些,比如推理与规划、近似可计算理论、正则化、提升理论、核方法、当然还有大名鼎鼎的统
\end{ulfield2}

\begin{ulfield2}[6]{主要参考文献}
第一部分,机器学习的底层理论:机器学习的底层理论有一些,比如推理与规划、近似可计算理论、正则化、提升理论、核方法、当然还有大名鼎鼎的统
\end{ulfield2}


\vspace{2.5em}
\begin{minipage}[b]{1\linewidth} 
\begin{tabular*}{0.2\linewidth}{lp{2cm}}
\quad  \\
\quad \\
(注:由指导教师填写)\\
\end{tabular*}
\hskip 3cm
\begin{tabular*}{0.5\linewidth}{lp{6cm}}
指导教师\myunderline[18]{}\hfill\\
接受设计(论文)任务日期\myunderline[10]{}\hfill\\
\hskip 8\ccwd 学生签名 \myunderline[10]{}\hfill\\ 
\end{tabular*}
 \end{minipage}
 
 
 
%============毕业设计(论文)考核评议书、毕业设计(论文)评审意见书==============%
 
 
% 这一部分的内容也可以用minipage来实现,或者用没有框的表格也行。谁来做一下。

 


\newpage

\parindent=2\ccwd

\begin{center}
{\ziju{1.5}\bfseries \LARGE西安交通大学}
\vspace{-\ccwd}
\end{center}

\begin{center}
\huge\bfseries
毕业设计(论文)考核评议书
\vspace{\ccwd}
\end{center}
\sectionmark{\small 毕业设计(论文)考核评议书}
\par\noindent\myunderline[10]{\@college}~院~\myunderline[10]{\@department}~系(专业)~\myunderline[8]{\@class}~班
\vspace{2\ccwd}

指导教师对学生\myunderline[10]{\@student}所完成课题为~\CJKunderline*{\@yourtitle}~的毕业设计(论文)进行的情况,
\begin{ulfield1}[7]{完成的质量及评分的意见:}

\end{ulfield1}

\vspace{\ccwd}
\begin{minipage}[t]{1\linewidth} 
\hskip 8.3cm
\begin{tabular*}{0.5\linewidth}{lp{5cm}}
指导教师\myunderline[11]{}\\
\hskip 2cm \quad 年\quad 月\quad 日\\
\end{tabular*}
 \end{minipage}



\begin{center}
\huge\bfseries
毕业设计(论文)评审意见书
\vspace{\ccwd}
\end{center}


\begin{ulfield1}[6]{评审意见:}

\end{ulfield1}

\vspace{\ccwd}
\begin{minipage}[t]{1\linewidth} 
\hskip 6cm
\begin{tabular*}{0.2\linewidth}{lp{6cm}}
评阅人\myunderline[8.5]{}& 职称\myunderline[6]{}  \\
 &  年  \quad 月 \quad 日\\
\end{tabular*}
 \end{minipage}
 
% 前面说了:
% 注意保留上面的\hfill},不要\hskip2\ccwd,但是在调用环境的时候在内容前加上\hskip2\ccwd(比如这一份表格的情况),或者不加(比如下个表格的情况)
% 就是指这里,不要开始两个字的缩进。

 
   
%===============毕业设计(论文)答辩结果==============%
\newpage
\parindent=2\ccwd

\begin{center}
  \quad  \\
  \quad    \\
{\huge\bfseries   毕业设计(论文)答辩结果 } \\
\vspace{\ccwd}
{ \bfseries  \LARGE    \myunderline[8]{}    院        }  \\                      
\hskip 3.5\ccwd{ \bfseries  \LARGE    \myunderline[8]{}    系(专业)   }\\
\end{center}


\sectionmark{\small 毕业设计(论文)答辩结果}

毕业设计(论文)答辩组对学生\myunderline[10]{}所完成
\begin{ulfield1}[3]{课题为}
\end{ulfield1}
\newline 的毕业设计(论文)经过答辩,
\begin{ulfield1}[7]{其意见为:}
\end{ulfield1}
\newline 并确定成绩为~\CJKunderline*{\hfill}

\vspace{6\ccwd}
\begin{minipage}[t]{1\linewidth} 
\hskip 5cm
\begin{tabular*}{0.2\linewidth}{lp{6cm}}
毕业设计(论文)答辩组负责人\myunderline[10]{}\\
答辩组成员:            \myunderline[8]{} \quad  \myunderline[8]{} \\

\hskip 5.8\ccwd     \myunderline[8]{}   \quad    \myunderline[8]{}  \\
\hskip 5.8\ccwd     \myunderline[8]{}   \quad    \myunderline[8]{}  \\
 \hskip 12\ccwd    年  \quad 月 \quad 日\\
\end{tabular*}
 \end{minipage}